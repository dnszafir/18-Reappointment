\documentclass[11pt]{article}

\usepackage{fancyhdr}
\usepackage[margin=1in]{geometry}
\usepackage{xcolor,colortbl}
\usepackage[labelfont=bf,font=small]{caption}


\usepackage[T1]{fontenc}

\usepackage{enumitem}
\usepackage{todonotes}
\usepackage{titlesec}
\usepackage{tabularx}
\usepackage{parskip}
\usepackage{hyperref}
\titlespacing\section{0pt}{0pt plus 2pt minus 2pt}{0pt plus 2pt minus 2pt}
\titlespacing\subsection{0pt}{12pt plus 4pt minus 2pt}{0pt plus 2pt minus 2pt}
\titlespacing\subsubsection{0pt}{12pt plus 4pt minus 2pt}{0pt plus 2pt minus 2pt}

\definecolor{Gray}{gray}{0.85}
\pagestyle{fancy}
\rfoot{\thepage}
\lfoot{\textbf{Service \& Outreach Statement} | D.A. Szafir}
\cfoot{}
\renewcommand{\headrulewidth}{0pt}
\renewcommand{\footrulewidth}{0.5pt}

%opening


\begin{document}
\setlength{\belowcaptionskip}{-10pt}
%\maketitle

\thispagestyle{fancy}

\textbf{\Large Service \& Outreach Statement}
{\hspace{145pt}\emph{Danielle Albers Szafir} \vspace{3pt}}
\hrule
%\\
%	{ 
%	\begin{tabularx}{\textwidth}{X X}
%		Assistant Professor & 315 UCB\\ 
%		Department of Information Science & University of Colorado Boulder \\ 
%		College of Media, Communication, \& Information &  Boulder, CO, 80309\\
%		University of Colorado Boulder & danielle.szafir@colorado.edu | 303.492.8532 
%	\end{tabularx}}}
%\hrule

As a member of a new Department and College, I have had a heavier service load than is traditional for a junior faculty member in my field. This service have in part been on curricular development committees (four total), which are discussed in detail in my Teaching Statement and excluded here. However, the explosive growth of data-oriented needs on campus has required substantial engagement in University committees given my role as the only visualization researcher at CU-Boulder and one of few women active in this area. My service and outreach has focused on providing that perspective to campus to help shape both personnel recruitment and support services offered. 

%I have additionally engaged in substantial professional service to my scholarly community. Much of this service has come in the form of participation on program committees. This role helps shape the technical program of major venues in my research discipline. My outreach efforts build on these service roles to promote diversity in data science and computing fields most notably through collaboration with the National Center for Women in Technology (NCWIT).

\section*{Summary of Accomplishments}

I have served on four departmental committees, one year as the Department's External Programs coordinator, one CMCI committee, and three campus committees. In addition to internal service, I have reviewed articles for 16 scholarly venues (95 total manuscripts) ranging in topic from visualization to big data analytics to virtual reality. I have served on nine program committees for major scholarly venues and three in-person grant review panels (two National Science Foundation panels and one Research Innovation Office panel), completing 25 total grant reviews. 
%Number of departmental, college, and university committees
%Number of paper reviews, number of PCs, number of grant reviews, number of panels
Beyond professional service, I have also engaged in public outreach through both participation in the CU Boulder Visualization Contest as a judge and through collaboration with NCWIT as a member of the Aspirations in Computing committee for the state of Colorado. 
%Outreach activities: Vis judge, ncwit aspirations

\section*{Department, College, \& University Service}
As a member of the founding faculty of the Department of Information Science, I have been heavily involved in establishing the curricular and cultural practices of our department. As such, I have twice served as a member of the faculty search committee, resulting in the successful hires of William Aspray, Lecia Barker, and Robin Burke. 

CU's prospective on Information Science is unlike many traditional iSchools, focusing on the intersection between people and technology in ways that advance our social and technical understanding of data and that lead to innovations at the intersection of computer, social, and data science. As such, we actively recruit Ph.D. students from multiple disciplines to support the broad variety of disciplinary skills and perspectives necessary to support this research. To support this work, I served as the Information Science representative on the Computer Science Graduate Committee, helping to bridge faculty in the disciplines to support recruiting and to help shape the role and structure of graduate computing education on campus. This role complemented my role as a member of the Information Science Graduate Committee. As a part of these committees, I helped with admission and recruitment events, reviewed Ph.D. student applications, routed applications and decisions to relevant faculty members, helped review new coursework and petitions, and helped with general review and revision of the programs as needs arose. I additionally served one year as our External Programs Coordinator, connecting INFO students and faculty to opportunities in other departments including Computer Science, ATLAS, Cognitive Science, and Communication and maintaining lists of related curricular and research offerings on campus. In this role, I frequently represented Information Science in meetings about on-campus initiatives and services. 

%My research orientation in data visualization has situated me to serve as a liason between Information Science and other departments on campus. As such, 
%external programs, both gradcomms

Because of my heavy engagement at the Department and University level, I have only served on one committee at the College level. I was a member of the Diversity and Inclusion Committee during the 2015 academic year, where I worked with a group of faculty from across CMCI to draft our policies on Inclusive Excellence and help shape how we could attract, support, and retain diverse students as part of the core mission of CMCI. 

At the University level, my scholarly orientation towards visualization and data science has led to membership on two committees focusing on research support for the increasing need for data services on campus: the Research Data Advisory Committee and the Advisory Board for the Center for Research Data \& Digital Scholarship (CRDDS). In these roles, I participate in discussions outlining the different needs for data-oriented scholarship across a broad variety of disciplines to help shape the services and resources offered to researchers campus-wide. I additionally had an opportunity to help shape data offerings on campus through participation on a faculty search committee in the Leeds School of Business, which led to the successful hire of David Eargle and by volunteering as a judge on the campus-wide data visualization contest offered through CRDDS. 

%University: Focus on expanding data science on campus. Only data visualization researcher means there's a big need for what I do. My curricular efforts are detailed in my Teaching Statement. 
%2017 - present Advisory Board Member, Center for Research Data \& Digital Scholarship (CRDDS)
%2017 - present Digital Humanities Certificate Committee Member
%2018 Visualization Contest Judge, Center for Research Data \& Digital Scholarship (CRDDS)
%2016 - 2017 Co-Chair, Digital Humanities Certificate Committee
%> Resulted in creation of a new interdisciplinary graduate certificate program
%2016 Faculty Search Committee, Leeds School of Business
%2015-2016 Research Data Advisory Committee

\section*{Professional \& Public Service}
I regularly serve as a reviewer for top publication venues in data visualization, computer graphics, and HCI. Since starting at CU, I have reviewed manuscripts for 16 venues, completing 95 reviews. 

In my research community, scholars are often expected to participate in Program Committees (PCs) that shape the technical program for conferences and symposia. Scholars are invited to the PC based on their prior successes and knowledge of the venue's research community. These responsibilities generally require reviewing 4-10 papers, recruiting external reviewers for each paper, leading discussions amongst reviewers to recommend acceptance or rejection, writing an additional review to summarize the strengths and weaknesses agreed on by reviewers, working with the papers chairs to determine ultimate acceptance or rejection, and supervising subsequent revisions. Service on a program committee requires on average 40 hours of work per PC. During my time at CU, I have served on nine PCs for seven academic venues and will serve on the PC for ACM CHI 2019. 

I have additionally completed 25 grant reviews, serving as an ad-hoc reviewer for the National Science Foundation (NSF) and Icelandic Research Foundation and participating in two review panels for the NSF and one for CU's Innovative Seed Grant program. Panel service requires attending in-person meetings to discuss and make recommendations on submitted proposals. For NSF panels, these meetings require travel to the NSF in Washington, D.C. for two days of discussion. 

In addition to referee service, I have served three years on the NCWIT Aspirations in Computing Awards Committee for the state of Colorado, a program recognizing high school women for outstanding achievements in computing. In this role, I work with Drs. Ken Anderson, Casey Fiesler, and Ben Shapiro and a team from NCWIT to review applications (283 total), select awardees and honorable mentions, and plan and deliver an awards ceremony, including a technical program and pre-program workshop for awardees. This program aims to increase women's representation in computing through recognition and community building. We have already seen direct evidence of these aims at CU: 12 2018 honorees (20\% of the total) formally accepted offers to pursue computing-related majors at CU for the 2019AY.

\section*{Current Service Trajectory}
To date, my service roles have fallen into three primary categories: (1) roles related to data science, (2) roles related to a new interdisciplinary department, and (3) professional service to my scholarly community. As Information Science continues to grow and committee roles become better established, I intend to refocus my service efforts to emphasize roles related to strengthening the research efforts of the Department and University with an explicit focus on roles related to the growth of data science support and personnel, especially in INFO's graduate enrollment. Similarly, I intend to refocus my professional service obligations to a smaller number of high-impact PCs allowing me to focus more directly on supporting the research community. 
%\todo[inline]{How can I say do less?}

\pagebreak
\setcounter{page}{1}


\end{document}
